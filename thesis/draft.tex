\documentclass[12pt]{article}
\usepackage{amsmath, amssymb, amsfonts}
\usepackage[margin=1in]{geometry}
\usepackage{setspace}
\usepackage{parskip}
\usepackage{fancyhdr}
\usepackage{graphicx}
\usepackage{subcaption}

\pagenumbering{arabic}
\linespread{1.5}
\setlength{\parskip}{12pt}
\pagestyle{fancy}
\fancyhf{}

\lhead{Ningrui Xie}
\chead{Thesis First Draft}
\rhead{ID: 400323416}
\fancyfoot[C]{\thepage}
\renewcommand{\headrulewidth}{0pt}
\newcommand{\perday}{\ensuremath{\mathrm{/day}}}

\begin{document}

\title{Mathematical Approaches for Simulating Epidemic Progression: Addressing Limitations of the Linear Chain Trick in ODE Models}

\author{Submitted to\\ Dr. Jonathan Dushoff 
\\McMaster University\\Hamiltion, Ontario, Canada L8S 4K1}
\date {March 22, 2024}
\maketitle


\centerline{Reported by}
\centerline{\textbf{Ningrui Xie}}
\centerline{\textbf{400323416}}


\newpage
\section{Abstract}

\section{Introduction}
Ordinary differential equations (ODEs) are extensively utilized for mathematical modeling in the biological sciences, offering a robust framework to accurately describe the temporal dynamics of systems. Their capacity to model continuous state changes makes them essential in capturing complex biological phenomena such as population dynamics \cite{salisbury2011mathematical}\cite{sego2021generation}, epidemiological patterns of infectious diseases \cite{Anderson1991}\cite{Diekmann2000}\cite{Feng2016}, and cellular processes like proliferation in oncogenesis \cite{baker1998modelling}\cite{jarrett2018mathematical}. The inherent flexibility and simplicity of ODEs make them not only easy to solve and understand but also easy to manipulate and accommodate to various situations.

In the field of epidemiology, ordinary differential equations (ODEs) serve as a crucial tool for characterizing disease progression and making predictions. During the COVID-19 pandemic, they were extensively utilized to guide policy-making, especially in disease control and public health interventions \cite{thompson2020epidemiological}. A commonly used model is the SIR compartmental model \cite{Anderson1991}\cite{kermack1927contribution}, which divides the population into three states: Susceptible $(S)$, Infectious $(I)$, and Recovered $(R)$. In the model, a susceptible individual, upon contact with the infected population at a rate $\beta$ $(\frac{1}{time})$, transitions immediately to the infectious state. An infected individual then moves to the recoverd state at a recovery rate of $\gamma$ $(\frac{1}{time})$. However, one major limitation of this model it its implicit assumption that the duration of the infectious stage follows an exponential distribution. This can be observed in a simplified equation that models the dynamics of the infected population. It assumes the entire population was infected at the outset, and individuals can only transition out of the infectious stage at a rate of $\gamma$:
\begin{align}
    \frac{dI}{dt} = -\gamma I
\end{align}
Solving this equation explicitly yields
\begin{align}
    I(t) = I_0 e^{-\gamma t}
\end{align}
where $I_0$ represent the initial infected population. Formula (2) implies a stochastic state transition model where the time spend in the infectious $I$ is exponentially distributed with a mean stage duration of $\frac{1}{\gamma}$. This dwell time distribution suggests that an individual has the highest probability of recovering from the disease immediately after infection, which is often not realistic. In reality, a delay between the successful infection of an individual and the recovery from the disease is inevitable. The work conducted by Byrne AW et al. \cite{byrne2020inferred} presents the simulation of parameter distribution inferred for the duration of the infectious period, and the shape of this distribution closely resembles a gamma distribution. The study by Greenhalgh and Day (2017) \cite{greenhalgh2017time}, utilizing measles outbreak data, also demonstrates that the recovery rate from an infectious disease is not static but can vary based on a wide range of factors.

It is important to note that even when the mean is constant, the shape of the stage duration distributions (i.e., the timing of stage transitions) can significantly impact the dynamics and the outcomes of predictive models in applied settings \cite{krylova2013effects}\cite{keeling2002understanding}\cite{wearing2005appropriate}\cite{nguyen2008noise}. Consequently, it becomes crucial to consider the type of distribution characterizing an individual in a specific state when constructing these models. A standard approach for developing models that incorporate different stage duration distributions is through the use of integro-differential equations (IDEs) or integral equations (IEs) \cite{hurtado2019generalizations}\cite{kermack1927contribution}\cite{hethcote1980integral}. IDEs facilitate the extension of the SIR model to include arbitrary distributions for the duration of infectiousness, as discussed in various sources \cite{feng2000endemic}\cite{hethcote1980integral}\cite{ma2006generality}. This approach allows for the capture of the 'memory effect', where the current rate of change in each compartment of the model (Susceptible, Infectious, Recovered) is influenced by the entire history of the epidemic, rather than relying on a constant rate assumption.

However, while integro-differential equations increase the model's flexibility, they present challenges in formulation, mathematical analysis, and simulation, particularly in comparison to ODEs \cite{krylova2013effects}\cite{hurtado2019generalizations}\cite{burton2005volterra}. An alternative method, the Linear Chain Trick (LCT), overcomes some of these challenges, offering a simpler and more direct approach to modifying the shape of distributions \cite{macdonald1978time}\cite{smith2011introduction}. This approach subdivides the single infected state into n substages, each characterized by an identical exponential distribution with a rate of $n\gamma$. This technique is based on the principle that the sum of a series of independent exponentially distributed random variables follows an Erlang distribution \cite{krylova2013effects}\cite{therrien2018probability}, which is a special case of the gamma distribution with an integer shape parameter. By employing this method, the distribution of the infectious stage duration transitions from exponential to Erlang. The central tendency of the Erlang distribution towards its mean value offers the model a more accurate representation of real-world scenarios.

When applying LCT into the SIR model, the number of substages $n$ determines the shape of the stage duration distribution. As $n$ increases, the variance decreases, resulting in a distribution that is more peaked and narrow around the mean value $\frac{1}{\gamma}$. As $n$ approaches infinity, the distribution converges to a delta function (a deterministic value, implying that all individuals who become infectious at time $t$ recover at exactly time $t + \frac{1}{\gamma}$), and the system evolves into a delay differential equation \cite{krylova2013effects}\cite{hethcote1980integral}.

The Linear Chain Trick (LCT) is extensively employed in modeling biological processes, not only in quantifying disease progression \cite{lloyd2001destabilization}\cite{lloyd2001realistic}, but also in analyzing virus \cite{lloyd2001dependence}\cite{kakizoe2015method} and population \cite{cushing2013integrodifferential} dynamics. Studies have demonstrated that switching the stage duration distribution from an exponential to an Erlang distribution leads to predictions that align more closely with real data points \cite{kakizoe2015method}. However, despite the increased realism they bring to modeling disease processes, employing LCT in ODE models presents several limitations. One primary challenge is the inflexibility of the model's parameters. To integrate independent exponentially distributed random variables into the ODE system, $n$ must be an integer. As mentioned above, since the shape of the stage duration distribution is determined by $n$, it would also be constrained to discrete changes. Another inherent limitation of using LCT is the cumbersome and time-consuming process of determining the appropriate number of substage to divide. Multiple models with varying substages must be constructed and subjected to repeated data-fitting processes to identify the optimized parameter value. Prior research has investigated approaches for estimating the number of states in the linear chain, employing techniques such as profile likelihood \cite{raue2009structural} and model reduction \cite{maiwald2016driving}. Yet, a comparative study of these methods indicates that the typical quality of experimental data is insufficient for reliably estimating the chain length using these approaches \cite{hauber2020estimating}. These limitations significantly restrict the adaptability and practical application performance of the model.

The recently proposed Generalized Linear Chain Trick (GLCT) enhances the Linear Chain Trick (LCT) by enabling models to incorporate a broader phase-type family of distributions \cite{hurtado2019generalizations}\cite{hurtado2021building}\cite{bladt2017phase}, which includes exponential, Erlang, hypoexponential, and Coxian distributions, thus allowing for greater flexibility and the ability to capture more complexity. The GLCT achieves this by creating a higher-dimensional Markov chain, enabling a more accurate representation of the complex nature of dwell time distributions. However, this process can be mathematically heavy and challenging to interpret. Additionally, this approach may lead to a increase in the number of variables or states within the model, potentially escalating computational demands. Therefore, finding a simpler and more efficient method to enhance the flexibility of the ODE model remains essential.

In this study, we focused on one particular direction: generalizing the Erlang distribution of stage durations currently present in the model. The goals were to have a more effective way for parameter estimation and to allow for a more diverse range of shapes, achieved by eliminating the constraints imposed by the shape parameter ($\kappa$). To address these limitations, we proposed the Geometric Chain Trick (GCT). Instead of having a variable number of substages $(n)$ with a constant transition rate between them $(\gamma_i = n\gamma)$, we fix $n$ to a specific integer value and use a series of rates that follow a geometric pattern. This series is represented by the formula $\gamma_i = ar^{i-1}$, governed by parameters $a$ and $r$. This approach enables a uniform model structure with the parameter of interest transitioning from discrete values $(n$ and $ng)$ to continuous values $(a$ and $r)$, thereby enhancing the model's adaptability (details in Section 4.x).

To illustrate this point, consider that the Gamma distribution can be parameterized in terms of its mean $(D)$ and shape parameter (quadratic coefficient of variation, $\kappa$), which determine the overall shape and position of the distribution. As mentioned above, in subdividing a single stage into n substages, each exponentially distributed substage must have a rate of $n\gamma$ to maintain the overall rate of that stage. While the mean of the Erlang distribution in the ODE system can be chosen freely, $\kappa$ is constrained to discrete values, shown as follows:
\begin{align*}
    Mean: \quad &\sum_{i=1}^{n_f} \frac{1}{n\gamma} = \frac{1}{\gamma}\\
    Variance: \quad &\sum_{i=1}^{n_f} \frac{1}{n^2\gamma^2} = \frac{1}{n\gamma^2} \\
    \kappa: \quad &\frac{\sigma^2}{M^2} = \frac{1}{n} \quad n \in \mathbb{N}
\end{align*}
Our approach, using a geometric series of rates, addresses this issue by allowing $\kappa$ to take any value within the range $\left[\ \frac{1}{n}, 1 \right)$. A detailed explanation and the derivation process are provided in section $4.2$. Another advantage of this method is that it enables a fixed model structure where only the parameters change, eliminating the need to determine optimal substage numbers through the construction of multiple models, a process currently required in the ODE model that employs the LCT.

Throughout this research, we primarily focused on the application of the LCT in the epidemic SIR model, which we refer to as the SI$^n$R model (or LCT model), where $n$ represents the number of substages divided. The objective of our study is to develop and evaluate the method of using a fixed model structure and a geometrically distributed substage rate as a replacement for the constant rate, aiming to enhance flexibility and efficiency. We have named the model that utilizes this method the GCT model. We anticipate that this approach will not only be computationally efficient but also provide a better fit to real-world data. In the following sections, we first analyze the model parameters by calculating the formulae for two essential properties: the mean and shape parameter. We then invert the process: given a desired $D$ and $\kappa$, we develop a computational system to determine the parameter values $(a and r)$ that enable the model to produce a stage duration distribution with these properties. The obtained parameter values are then used to compare the GCT model with the LCT model through simulation. Subsequently, we conduct empirical validation of our model through data fitting. This fitting confirmed that the GCT model enables the use of a uniform model structure with enhanced flexibility and practicality, thus paving the way for the development of user-friendly software tools that empower modelers to efficiently select and employ the most suitable models for describing and predicting future epidemic scenarios.


\section{Method}
Several assumptions have been made for the unforced SIR model (Susceptible, Infectious, and Recovered) that our study is based on: 
\begin{enumerate}
    \item \textbf{Closed Population}: A constant population size with equal birth and death rates and no disease-related mortalities.
    \item \textbf{Homogeneous mixing}: Individuals have an equal chance of coming into contact with any other individual in the population. 
    \item \textbf{Ignoring Latency}: Individuals move immediately from a susceptible state to an infected state, ignoring any incubation period associated with the disease.
    \item \textbf{Lifelong immunity}: Once an individual recovers, they do not become susceptible or infectious again.
    \item \textbf{Fixed rate of transmission and recovery}: The rate of disease spread and the rate at which infected individuals recover and become immune are constant over time.
\end{enumerate}

The dynamic of the population within the three compartment can be expressed as a system of nonlinear ordinary differential equations:
\begin{align}
    \frac{dS}{dt} &= \mu N - \frac{\beta SI}{N} - \mu S \\
    \frac{dI}{dt} &= \frac{\beta SI}{N} - \gamma I - \mu I \\
    \frac{dR}{dt} &= \gamma I - \mu R
\end{align}
where $S$, $I$, and $R$ represent the susceptible, recovered, and infectious population. $N$ is the size of the total population, and in this case $S + I + R = N$. The $\mu$, $\beta$ and $\gamma$ are the rates of natural birth (death), transmission and recovery, respectively. $\beta$ is the rate of new infections per unit time, so $\beta SI$ is the number of new infections that occur per unit time (the incidence rate). Here, the mean recovery time for an infected individual is $\frac{1}{\gamma}$. By assuming that the entire population was infected at the onset and considering a zero natural birth (death) rate, we can focus solely on the dynamics of the recovered population, as indicated by formula (1). 

The idea of dividing the infectious stage into $n$ sub-stages exploits the fact that the sum of a sequence of independent, exponentially distributed random variables follows an Erlang distribution (i.e. gamma distributed with integer shape parameter) \cite{therrien2018probability}
\begin{equation}
    f(x; n,n\gamma) = \frac{(n\gamma)^n}{(n-1)!} x^{n-1} e^{-n\gamma x} \quad x>0, n \in \mathbb{N}
\end{equation}
Utilizing the principle of the Linear Chain Trick (LCT), we divide the single infectious stage into $n$ substages, each with a constant rate of $n\gamma$. Therefore, formula (1) will be replaced by the following set of equations, which we refer to as the cohort model:
\begin{align}
    \frac{dI_1}{dt} &= - n \gamma I_1 \\
    \frac{dI_2}{dt} &= n\gamma I_1 - n \gamma I_2 \\
    &\vdots \\
    \frac{dI_n}{dt} &= n\gamma I_{n-1} - n \gamma I_n \\
    \frac{dR}{dt} &= n \gamma I_n
\end{align}
Note that this decomposition process will not change the mean stage duration, as the sum of the means of the $n$ exponentially distributed random variables still equals $\frac{1}{\gamma}$
\begin{align*}
    \sum_{i=1}^{n} \frac{1}{n\gamma} = \frac{1}{\gamma}
\end{align*}

The Geometric Chain Trick (GCT), on the other hand, features a fixed number of substages along with a set of inter-substage transition rates characterized by a geometric series, rather than a constant rate, as illustrated below:
\begin{align}
    \frac{dI_1}{dt} &= - a I_1 \\
    \frac{dI_2}{dt} &= a I_1 - ar I_2 \\
    \frac{dI_3}{dt} &= ar I_2 - ar^2 I_3 \\
    &\vdots \\
    \frac{dI_n}{dt} &= ar^{n-2} I_{n-1} - ar^{n-1} I_n \\
    \frac{dR}{dt} &= ar^{n-1} I_n
\end{align}

Having established the model, our initial step involved analyzing the parameters within the GCT model. We aim to determine whether employing a fixed model structure (i.e., a fixed number of substages) and manipulating the parameters $a$ and $r$, which define the geometric transition rate, can replicate the effects of varying the number of substages. Additionally, we aim to evaluate whether the GCT structure offers greater flexibility in the choice of $\kappa$ $(CV^2)$. By calculating the mean $(D)$, variance $(\sigma)$, and shape parameter $(\kappa)$, we are able to represent all these properties of the infectious stage duration distribution in terms of $a$ and $r$. Similar to the LCT, utilizing the GCT facilitates independent alterations in the mean and $\kappa$. Given this, we can invert the process: starting with specific values of $D$ and $\kappa$, we can identify the corresponding $a$ and $r$ that endow the model with such characteristics. However, due to the complexity of the formula, solving it explicitly is challenging. To overcome this, we have developed a system of functions in the \textbf{R} programming language, employing \verb|uniroot()| to determine the values of $a$ and $r$. These parameters are then integrated to the GCT model to compare simulation outcomes with those of the LCT model, examining alignment in both the infectious stage duration distribution and the incidence over time.



\section*{Results}

\section{Discussion}


%%%%%%%%


This is a citation \cite{Feng2016}

Another citation \cite{Anderson1991} 

Here's a simple plot (with no dependence)

\begin{figure}[h!]
    \centering
    \includegraphics{figures/mytest1.Rout.pdf}
    \caption{Mytest1 plot (are you working??)}
\end{figure}


Another plot (depend on tempfunc.R)

\begin{figure}[h]
    \centering
    \includegraphics{figures/mytest2.Rout.pdf}
    \caption{Mytest2 plot}
\end{figure}

Maybe another plot from a different folder:

\begin{figure}[h]
    \centering
    \includegraphics{figures/test.Rout.pdf}
    \caption{test plot}
\end{figure}





\bibliographystyle{unsrt}
\bibliography{Reference} 

\end{document}
